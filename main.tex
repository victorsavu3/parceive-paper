\documentclass[conference]{IEEEtran}

\ifCLASSOPTIONcompsoc
  \usepackage[nocompress]{cite}
\else
  \usepackage{cite}
\fi

\usepackage[pdftex]{graphicx}
\graphicspath{{../img/}}

\usepackage{color}
\input{macros}

\usepackage[T1]{fontenc}
\usepackage[scaled]{inconsolata}

\definecolor{bluekeywords}{rgb}{0.13,0.13,1}
\definecolor{greencomments}{rgb}{0,0.5,0}
\definecolor{redstrings}{rgb}{0.9,0,0}
\definecolor{mygray}{rgb}{0.5,0.5,0.5}

\usepackage{listings}
\lstset{language=C,
showspaces=false,
showtabs=false,
breaklines=true,
captionpos=b,
frame=single,                    % adds a frame around the code
showstringspaces=false,
breakatwhitespace=true,
escapeinside={(*@}{@*)},
commentstyle=\color{greencomments},
keywordstyle=\color{bluekeywords}\bfseries,
stringstyle=\color{redstrings},
basicstyle=\ttfamily\footnotesize,
morekeywords={NUM_SUCCESSORS, NUM_GROUPS},
title=\lstname
}

\begin{document}

\title{A Visualization Framework for Parallelization}

\author{A. Wilhelm,
		V. Savu,
		E. Amadasun,
		T. Schuele,
        and~M. Gerndt}% <-this % stops a space

\author{
	\IEEEauthorblockN{
		Andreas Wilhelm, 
		Victor Savu,
		Efe Amadasun,
		Michael Gerndt \\ 
		\IEEEauthorblockA{
			Technische Unversit\"at M\"unchen\\
			\{wilhelma, savu, amadasun, gerndt\}@in.tum.de
		}
		\and
		Tobias Schuele \\
		\IEEEauthorblockA{
			Siemens Corporate Technology\\%
			tobias.schuele@siemens.com
		}
	}
}

\IEEEtitleabstractindextext{%
\begin{abstract}
Since the advent of multicore processors, developers struggle with the
parallelization of legacy software. Automatic methods are only appropriate to
identify parallelism at instruction level or within simple loops. For most
applications, however, a scalable refactoring requires profound comprehension
of the underlying software architecture and its dynamic aspects. This leads to
an increasing demand for interactive tools that foster parallelization at
various granularity levels. To cope with this problem, we propose a
visualization framework and three tailored views for parallelism detection. The
framework is part of \emph{Parceive}, a tool that utilizes dynamic binary
instrumentation to trace C/C++ and C\# programs. The cooperative views allow
identification and analysis of potential parallelism scenarios using seamless
navigation, abstraction, and filtering. In this paper, we motivate our
approach, illustrate the architecture of the visualization framework, and
highlight the key features of the views. A case study demonstrates the
usefulness of Parceive.
\end{abstract}
\begin{IEEEkeywords}
Parallelization, Trace analysis, Software visualization, Program comprehension.
\end{IEEEkeywords}
}

\maketitle

\IEEEdisplaynontitleabstractindextext
\IEEEpeerreviewmaketitle

\section{Introduction}
\label{sec:introduction}
Multicore processors still poses a big challenge to the industry and the
research community. To exploit the increasing processing power of such
processors software parallelism is indispensable. As a result, software vendors
are forced to parallelize their legacy software to make them scalable. Such a
parallelization relies on two essential steps. The first step is to
sufficiently comprehend the given software design and its dynamic behaviour.
The second step is a potential redesign of the software architecture that
enables parallelism. Unfortunately, both steps are tedious and error-prone so
that most of the legacy software still operates sequentially. Besides the
additional complexity of parallel programming, we believe the problem largely
arises due to a lack of appropriate tool support.

In the field of reverse engineering, software analysis tools are invaluable to
understand the structural and behavioral aspects of software systems. These
tools gather necessary data using static (at compilation time) or dynamic
(at runtime) analysis approaches and effectively visualize this data. Dynamic
analysis yields precise information about concrete runtime events. Static
Analysis conservatively reasons over possible behaviors by examining system
artifacts, e.g., source code. It has been cited that hybrid approaches are well
suited to provide good accuracy and soundness of the
analyses~\cite{StaticDynamic}. We implemented Parceive, a tool that combines
static and dynamic analyses to provide effective vies for
parallelization~\cite{Parceive}.

In this paper, we present a visualization framework and two views for Parceive.
Our framework enables efficient analyses on large traces to answer elaborate
user queries. We further support an easy integration of tailored views by
providing a common interface to them. This allows us to address a variety of
viewpoints without writing highly specific analyses. Various optimization and
abstraction techniques in the visualization framework ensure responsiveness and
scalability, e.g., by on-demand loading, caching, trace abstraction, or
communication across views. The two views we present enable users to detect
hotspots, infer parallelization strategies, and validate these strategies
regarding data dependencies.
\section{Parceive}
\label{sec:parceive}
%We implemented Parceive, a tracing-based tool for interactive software
%analysis~\cite{Parceive}. 
The vision behind Parceive is to help developers in identifying parallelism
opportunities and obstacles at various granularity levels. It utilizes static
binary analysis and dynamic instrumentation to collect trace data. Being less
conservative than purely static approaches lets us focus on concurrency-related
events, e.g., memory accesses, routine invocations, and object instantiations.
By a-posteriori abstraction of such fine-grained information, we infer
architectural aspects from the applications. 
The results can be used as a starting point for architecture redesign and
refactoring. However, due to the inherent incompleteness of dynamic analysis,
the user is responsible for correct parallelization.

\begin{figure}[h!]
	\begin{center}
		\includegraphics[width=0.40\textwidth]{img/parceive}
		\caption{The high-level components of Parceive.}
		\label{fig:parceive_overview}
	\end{center}
\end{figure}

Figure \ref{fig:parceive_overview} depicts the fundamental components of
Parceive and their relations. Our analyses operate on executables to retrieve
the required information. These analyses can be classified into static and
dynamic ones. The former inspect the data- and control-flow of single linkage
objects. By incorporating debug symbols, static analysis enables our tool to
gather information about variable accesses, loop constructs, and class
hierarchies. Additionally, the gathered information is used to restrict the
scope of subsequent runtime analysis in order to reduce the execution overhead.

The runtime analysis instruments and inspects predefined events during
execution of an application, e.g., object instantiations, method invocations,
and thread handling (in case of multi-threaded applications). During such
events, Parceive collects trace data and stores it in an SQL database. The
database scheme enables highly specific and performant queries for different
visualizations which are key for program comprehension and parallelization.
Each view thereby simplifies and highlights specific aspects of the traced
software.
\section{Visualization Infrastructure}
In this section, we present certain key aspects of our web-based visualization
infrastructure. This infrastructure facilitates the integration of arbitrary
views to Parceive by providing a common interface for accessing abstracted
runtime traces. The biggest challenge when dealing with traces is the
potentially overwhelming amount of data. This often leads to unmanageable views
with unacceptable delays. Our visualization infrastructure addresses this
problem by building upon a reactive client-server architecture (see Figure
\ref{fig:visualization}) that provides four key services: (a) trace
optimization, (b) on-demand loading, (c) caching and pipelining, and (d)
state management and communication.

\begin{figure}[h!]
\includegraphics[width=\linewidth]{img/visualization_framework}
\caption{The visualization infrastructure of Parceive.}
\label{fig:visualization}
\end{figure}

\paragraph{Trace Optimization}
As mentioned previously, Parceive uses a database to store trace data. Its
layout is tuned for write operations to reduce the runtime overhead. All the
information required by the views can be obtained via database queries.
However, most of the read operations would take a disproportionate amount of
time to complete without applying certain optimizations on the trace database.
Hence, we perform a set of optimizations during a post-processing step to
reduce lookup times for views.

The most important optimization is the generation of table indices for
efficiently searching the database. Another optimization is the creation of
intermediate tables to avoid expensive joins for most queries. Although this
leads to some redundancy, it does not increase the overall complexity.
Furthermore, reducing fragmentation within tables increases data locality and
thus speeds up queries. This has considerable effect on queries that require a
full table scan and also reduces the size of the trace database.

\paragraph{On-Demand Loading}
On-demand loading of trace data improves responsiveness of the visualization.
Often, loading entire traces into the browser is not feasible due to memory
restrictions. To solve this problem, a NodeJS server has been developed that
reads data on demand. The server provides a REST~\cite{rest} API to manage
retrieval of data. For security reasons, all SQL queries are performed by the
server without supporting arbitrary queries. The implementation makes use of
multiple parallel reads to the database to reduce the latency and to increase
the throughput when large amounts of data are requested by the views. This lets
users seamlessly explore and navigate through software producing a potentially
unmanageable amount of trace data.

\paragraph{Caching and Pipelining}
At the client side, we provide an Object Relational Mapper (ORM) module to
simplify development and improve responsiveness. This module enables accesses
to predefined entities and manages the relationships between them. The API is
implemented using promises~\cite{promises} which simplify asynchronous and
parallel access. The greatest benefit for views using the ORM are optimizations
for data loading, the most important ones being caching and pipelining. Caching
avoids repeated loading of data that was accessed before, and pipelining
combines multiple queries to the same endpoint into a single one. When
requesting a large number of entities, pipelining heavily improves the
throughput with only small latency overhead.

\paragraph{State Management and Communication}
The visualization framework provides global state management and communication
facilities to views. The former is based on a centralized and persistent state
storage that retains the state of views across page loads. Currently,
the view layout and the marked nodes are automatically stored as part of the
state. In addition, each view can save tailored information at any time and
retrieve it during rendering. Local storage is used to keep all the state
information making it persistent. This service reduces computational effort for
views by reusing results across arbitrary UI events.

The communication service enables arbitrary views to interact by triggering
predefined events. These events let users explore their applications with
synchronized representations from complementary viewpoints. One example is
simultaneous highlighting of entities such as functions in different views.
Another example is spotting of distinct entities for further inspection in
separate views. This way, the number of nodes to be displayed in a view is
reduced which increases scalability. Currently there the following types of
communication provided:

\begin{itemize}
	\item \textit{Focusing} brings entities to the attention of the user by
centering the representations in all views.
	\item \textit{Marking} allows to communicate selected entities between views.
	\item \textit{Spotting} replaces the shown entities in views by a new
collection of entities.
	\item \textit{Hovering} highlights entities in multiple views by reducing the
opacity of all other entities.
\end{itemize}
\section{Views}
In this section, we present three prototypical views built upon the
presented visualization infrastructure. The interaction of the views provide a
scalable top-down approach for identifying hotspots, developing parallelization
strategies, and validating these strategies with respect to data dependencies.

\subsection{Performance View}
\label{sec:performance_view}
The performance view is an interactive representation of a program's profiling
and trace data (see Figure \ref{fig:emsim}). Its primary purpose is to assist
users in identifying scenarios that may benefit from parallelization. A
holistic visualization of a program's runtime behavior helps users to spot
potential performance and guides optimization efforts.

There are two visualization modes in the performance view: tracing and
profiling. The tracing mode represents an icicle plot augmented with loop
information, i.e., a hierarchical view of calls made during program execution.
The calls are arranged chronologically from left to right, and the visibility
of loop information can be toggled (on or off). Trace visualization is useful
for detailed examination of a program, especially when the traced invocation
order is important. The profiling mode presents the user a hierarchical view of
the functions called during execution. The length of each function in the view
is given by the sum of the calls' execution time, which is often sufficient to
quickly pinpoint hot spots.

As a program's size and complexity increases, so does the quantity of its
performance data, making the data harder to digest at once. For this reason,
the performance view provides zooming at call level and execution time
filtering. The latter sets the minimum execution time required for a call to be
loaded. The minimum value depends on the percentage of the execution time of
the current top-level call in the view. This only allows the rendering of calls
coarse enough to be easily visible. With the call zooming feature, users can
focus on a specific call, which makes it the top-level call, recomputes a new
minimum value, and loads child calls of the focused call that weren't visible
before.

\subsection{Calling Context Tree (CCT) View}
The CCT view targets comprehension of the dynamic behavior of an application.
It displays a calling context tree consisting of call nodes, loop nodes, and
memory nodes (see Figure \ref{fig:cct_view}). Nodes for calls (or groups of
calls), loop executions, and loop iterations are positioned using a tidy tree
layout~\cite{TidierTree}. Children of nodes are vertically sorted by their
start time to reflect their actual execution. Memory nodes accessed by
arbitrary tree nodes are difficult to integrate in a tree layout. Therefore,
they are positioned using an unconstrained layout based on a force simulation
around the rest of the tree.

\begin{figure}[h!]
\includegraphics[clip, trim=0.9cm 2.5cm 8.5cm 8.5cm,
width=\linewidth]{img/cct_view}
\caption{The CCT view of Parceive showing function calls (rectangular nodes),
loops, and accessed memory references (circular nodes).}
\label{fig:cct_view}	
\end{figure}

The first node present when the view is created is the call to the
\texttt{main} function. Users can arbitrarily expand and collapse call nodes.
When a function is called multiple times during the same function execution,
the respective call nodes are merged to so-called \textit{call groups}. Call
groups reduce the number of nodes to be displayed but can also be decomposed
into their single call nodes. Navigating through loop executions and loop
iterations is similar to calls and allows the user to see information at any
desired granularity. 

The most important use-case of the CCT view for parallelization is data
dependency analysis. Users are often interested in the existence and the
location of such dependencies between arbitrary regions of their software.
Therefore, optimized queries of the visualization infrastructure are utilized
to detect shared memory accesses across deep call hierarchies. Found
dependencies can then be inspected in a separate source code view. The
described feature allows manageable visualization by dramatically reducing the
amount of shown nodes. There are some additional features that aim for better
scalability and help developers parallelizing their code:

\begin{itemize}
	\item Profiling information (relative execution time) is reflected by node
colors.
	\item Expanding and collapsing nodes can be performed in both
directions to show and hide parent or child nodes.
	\item Seamless zooming or panning, and focusing on single entity nodes for
a clear visualization.
	\item Spotting of arbitrary call nodes, which automatically expands the call
tree so that it ends with these nodes and starts with their common ancestor.
\end{itemize}

\subsection{Source View}
The source view shows provided source code of the instrumented application. The
usefulness of this view becomes apparent when it is communicating with the
other views presented in this paper. The simplest form of interaction is
focusing, where the source view displays functions, loops, and memory accesses.
Focusing makes it easy to follow the execution of a program trough the source
code. Hovering provides additional information; for calls it indicates where
the call originated, and for memory references where they were allocated and
referenced.
\section{Case Study: EMSim}
\label{sec:case_study}
In this section, we demonstrate Parceive by applying it on EMSim, a simulator
for the european soccer championship. EMSim was part of the student assignments
for our Master course on parallel programming. The application simulates every
match of the group and final phase by relying on historical statistics of
matches, teams, and players. The task was to parallelize the sequential
C code, by considering the following steps: (1) the comprehension of EMSim and
its behaviour, (2) the location of hotspots, (3) the identification of
dependencies, and (4) the parallelization of potential code regions with
appropriate libraries. When done, the students were able to upload their source
code on our submission server to check for correctness and speedup. As it
turned out, the biggest challenges were to estimate the load balance and to
identify data dependencies. In the following, we show how Parceive would have
helped the students with both issues.

In listing \ref{listing:playEM}, we present an excerpt of the EMSim code that
contains a major region for parallelism. The shown \texttt{playEM()} function
gets multiple arrays of teams as input, one for each group of the championship.
Herein, the loop in line 7 calls \texttt{playGroup()} for every group, which
plays all group matches between the given teams and returns the first, second,
and third winners. This loop accounts for \textasciitilde60\% of the overall
runtime of an execution. Even though students might spot the loop as a
potential opportunity for parallelism, the crux is to analyze dependencies
between the calls. The first challenge is to understand the pointer arithmetic
in line 10 and 11, which scatter qualified teams to the successor array. The
second challenge is to inspect the (arbitrary deep) call hierarchies for shared
memory accesses.

\begin{lstlisting}[caption=A major region for loop-parallelism in the
EMSim code, label=listing:playEM]
  // play groups
  for (int g = 0; g < NUM_GROUPS; ++g) {
    playGroup(
      teams + g,
      successors + (g * 2),
      successors + (NUM_SUCCESSORS - (g * 2) - 1),
      bestThirds + g
    );
  }
\end{lstlisting}

We applied Parceive on EMSim by tracing the optimized user binaries (gcc
compiler flag \texttt{-og}). Other libraries that are used by EMSim were not
instrumented since we can preclude conflicting accesses with the user
application. The instrumentation framework and the analyses cause an execution
slowdown of factor \textasciitilde 4.4 (20s instead of 4,5s). The produced
trace data has a size of 4.3Mb, where data about memory accesses and function
invocations account for 84\% of the size. After producing the trace database,
we can import it to our visualization and show the resulting views.

The trace view that depicts the EMSim execution is shown on the upper part
of Figure \ref{fig:emsim}. By doing a top-down inspection, it becomes obvious
that \texttt{main()} spends most of the time in the invocation of
\texttt{playEM()}. Within this function call, the execution time is mainly
spent in two loops. The first loop (the one from Listing \ref{listing:playEM})
performs the group phase. The second loop plays the final rounds, starting from
the round of 16 to the final match.

Recall the two main purposes of the trace view, identifying hotspots
and estimating load balance. In the case of EMSim, the view indicates the
invocations of \texttt{playGroup()} and \texttt{playFinalRound} as significant
hotspots. Without considering dependencies, a naive strategy could invoke the
single calls by individual threads. Besides limited scalability (there are only
6 calls of \texttt{playGroup()}), this solution results in a non-optimal load
balance since the calls vary in their execution time. But the trace view
exposes another opportunity for parallelism. Both the group matches and the
final matches lead to distinct calls of \texttt{playMatchGeneral()}. Hence,
a second strategy is to use a master-worker pattern for scheduling these calls
tron a thread-pool. The next step for the user is to validate the found
parallelization strategies.

The CCT view supports users with the validation of prospective system
redesigns. We present the resulting view for EMSim in Figure \ref{fig:emsim}.
Notice that only relevant call nodes for the investigated parallelization
strategies are shown by spotting them on the trace view. Beginning with the
first strategy, we are interested in possible data dependencies between the
invocations of \texttt{playGroup()} and \texttt{playFinalMatch()},
respectively. The corresponding analysis (\textit{show deep dependencies})
queries shared accesses among full call hierarchies. It turns out that each
invocation of \texttt{playGroup()} accesses the global variable
\texttt{groupGoals} and each of invocation of \texttt{playFinalMatch()}
accesses the stack variables \texttt{goals1} and \texttt{goals2}.
Furthermore, the cct view enables to localize these memory accesses such that
the dependencies can be investigated and synchronized.


Even though the data dependencies for the first parallelization strategy
can be easily avoided, we focus on the other strategy that put the calls to
\texttt{playMatchGeneral()} in distinct tasks. The respective CCT view is shown
in Figure \ref{fig:emsim}. Performing the same dependency analysis as
before returns no shared memory accesses between the calls. Hence, the user
gained the necessary information to parallelize EMSim by integrating the
master-worker pattern. A student solution achieved a speedup of 6.5 on our
8-core system. Considering the sequential pre- and post-processing regions,
and the control dependent matches of the final phase (e.g., the semi-final has 
to be played before the final), this speedup confirms Amdahl's law. This result
concludes that our views are very helpful to parallelize user applications.

\section{Related Work}
\label{sec:related_work}
\todo{Write related work}
\section{Future Work}
\label{sec:future_work}
Parceive was developed using a data-centric approach, where the initial project
phase focused on fine-grained trace data and its visualization. In the second
project phase, we will apply abstraction on this data to gain architectural
insights for software comprehension and parallelization. Additional knowledge
from software architects and developers will be incorporated to specify
software components. This knowledge can then be used for extended views that
depict the structure of systems and their behaviour on architecture level.

Currently, we are working on multiple features that improve the usability of
Parceive, or even extend possible uses. One such feature is a source code view
that can focus on arbitrary regions of user code. This view allows to link
nodes from any visualization to the respective definitions or usages. Another
feature we are working on are views that depict parallel behavior of
multi-threaded applications. Possible use-cases are scalability analysis or
validation, e.g., to identify possible race conditions.

We reviewed the visualization component on a few open and closed source
applications, some of them are at industrial scale. The tests focused on the
scalability of the framework and the effectiveness of the views. However, more
tests necessary to measure performance and responsiveness, and to elicit
possible scaling problems. We plan to focus on an extensive set of sequential
user applications with decent parallelism potential.
\section{Conclusion}
\label{sec:conclusion}
Visualization of software behavior is a crucial instrument to comprehend
software systems. Unfortunately, this task often involves an overwhelming
amount of trace data that easily overstrains current UIs. Therefore, we built a
highly scalable and responsive visualization infrastructure for our tracing
tool Parceive. The core is a client-server architecture providing asynchronous
event driven data retrieval and optimized analysis queries. These techniques
empower users to explore and navigate their traced applications by multiple
viewpoints. This paper additionally presented three views and a use-case that
demonstrates the usefulness of these views on identification and validation of
potential parallelism scenarios. Currently, we are in an early stage of
research. Future work will mainly cover higher-level aspects to comprehend
software systems in a top-down approach.
%
\section*{Acknowledgements}
We thank Nathaniel Knapp for comments and suggestions. Additionally, we
gratefully acknowledge the support of Siemens. This work has been supported by
the German Federal Ministry of Education and Research (Software Campus, grant
no. 01IS12051).
%

\bibliographystyle{abbrv}
\bibliography{biblio}

\end{document}


