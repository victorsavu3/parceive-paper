\section{Future Work}
\label{sec:future_work}
Parceive was developed using a data-centric approach, where the initial project
phase focused on fine-grained trace data and its visualization. In the second
project phase, we will apply abstraction on this data to gain architectural
insights for software comprehension and parallelization. Additional knowledge
from software architects and developers will be incorporated to specify
software components. This knowledge can then be used for extended views that
depict the structure of systems and their behaviour on architecture level.

Currently, we are working on multiple features that improve the usability of
Parceive, or even extend possible uses. One such feature is a source code view
that can focus on arbitrary regions of user code. This view allows to link
nodes from any visualization to the respective definitions or usages. Another
feature we are working on are views that depict parallel behavior of
multi-threaded applications. Possible use-cases are scalability analysis or
validation, e.g., to identify possible race conditions.

We reviewed the visualization component on a few open and closed source
applications, some of them are at industrial scale. Main attention has been paid
to the scalability of the framework and to the effectiveness of the views.
However, there are more tests necessary to measure performance and
responsiveness, and to elicit possible scaling problems. We plan to focus on an
extensive set of sequential user applications with decent parallelism
potential.