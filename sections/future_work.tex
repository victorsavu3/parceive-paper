\section{Future Work}
\label{sec:future_work}
Parceive was developed using a data-centric approach, where the initial project
phase focused on fine-grained trace data and its visualization. In the second
project phase, we will apply abstraction on this data to gain architectural
insights for software comprehension and parallelization. Additional knowledge
from software architects and developers will be incorporated to specify
software components. This knowledge can then be used for extended views that
depict the structure of systems and their behaviour on architecture level.

Currently, our visualization is lacking some capabilities to precisely
interpret the view contents. One such feature is the relation between view
nodes and the corresponding regions in the user source code. Hence, we will
provide a source code view and enable to focus on arbitrary regions. A concrete
example is the inspection of data dependencies, where the user can focus on
the memory access instructions in the code. Another missing feature is the
support of already multi-threaded applications. Possible uses-cases are
scalability analyses or validations, e.g., to identify possible race
conditions. We plan to integrate these features in the next release.

We reviewed the visualization framework and the presented views by applying 
them on a few open and closed source applications. Main attention has been paid
to the scalability of the framework and to the effectiveness of the views.
However, there are more tests with software at industrial scale necessary to
elicit possible scaling problems. We plan to focus on an extensive set of
sequential user applications with decent parallelism potential.
