\section{Related Work}
\label{sec:related_work}
Trace visualization has a long history in analyzing software
parallelism~\cite{TraceVisualization}. Current visualizations are able to
handle up to $10^7$ events or terabytes of trace data~\cite{State}. However,
most of the cited methods achieve these numbers only for statistical plots or
by averaging pixels. Others gain scalable views at the expense of limited
usability, e.g., by requiring users to extensively pan across detailed trace
data. Our tool tries to create the sweet spot between full abstraction and
largely unprocessed details through a more interactive approach.

Vampir and HPCToolkit are tools from the performance community that provide
visualization environments to represent performance data~\cite{Vampir,
HPCToolkit}. This enables a detailed understanding of dynamic processes on
massively parallel systems. Vampir handles terabytes of data via parallel1
preprocessing and data reduction techniques during collection. HPCToolkit
maintains interactivity of its views by sampling the data rather than on-demand
processing during panning and zooming. Both tools focus on parallel software to
highlight performance and communication issues. In contrast, the intention of
Parceive is to aid with parallelization of legacy applications.

Moose certainly belongs to the most prominent reverse engineering
frameworks~\cite{Moose}. It provides a set of services including a common
meta-model, metric evaluations, a model repository, and generic GUI support for
querying, browsing and grouping. Metric results can arbitrarily visualized by a
set of views, e.g., polymetric views, evolution matrices, or butterfly views.
Though the functionality of Moose by far exceeds the one of our tool, we often
rely on highly optimized search strategies (e.g., deep dependency analysis) and
specialized views for parallelization.