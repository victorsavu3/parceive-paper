\section{Case Study: EMSim}
\label{sec:case_study}
In this section, we demonstrate Parceive by applying it to EMSim, a simulator
for the European soccer championship. EMSim was part of the student assignments
for a master course on parallel programming. The application simulates all
matches by relying on historical statistics of results, teams, and players. The
task was to parallelize the sequential C code by considering the following
steps: (1) the comprehension of EMSim and its behavior, (2) the location of
hotspots, (3) the identification of dependencies, and (4) the parallelization
with appropriate libraries. The students were able to upload their source code
to a submission server for checking correctness and speedup. As it turned out,
the biggest challenges were to estimate the load balance and to identify data
dependencies. In the following, we show that Parceive helps with both issues.

\begin{lstlisting}[caption=A major region for loop-parallelism in the
EMSim code, label=listing:playEM]
  // play groups
  for (int g = 0; g < NUM_GROUPS; ++g) {
    playGroup(
      teams + g,
      successors + (g * 2),
      successors + (NUM_SUCCESSORS - (g * 2) - 1),
      bestThirds + g
    );
  }
\end{lstlisting}

Listing \ref{listing:playEM} is an excerpt of the EMSim code that contains a
major opportunity for parallelism. The shown loop is contained in the
\texttt{playEM()} function, which gets arrays of teams as input. Every
iteration calls \texttt{playGroup()} for each group of the preliminary round,
which simulates the matches between the given teams and returns the first,
second, and third winning team. This loop accounts for \textasciitilde60\% of
the overall runtime. Even though the loop is easily identified as an
opportunity for parallelization, the crux is to analyze dependencies between
the calls. The first challenge is to understand the pointer arithmetic in lines
5 and 6, which scatters qualified teams to the successor array. The second
challenge is to inspect the (arbitrarily deep) call paths for accesses to
common memory locations.

We applied Parceive to EMSim by tracing an execution of the optimized user
binaries. The instrumentation framework and the runtime analyses caused an
execution slowdown factor of \textasciitilde 4.4 (20s instead of 4,5s). The
produced trace data had a size of 4.3Mb, where data about memory accesses and
function invocations account for 84\% of the size. After producing the trace
database, we imported it to our visualization infrastructure.

The trace view that depicts the execution of EMSim is shown in Figure
\ref{fig:emsim}. The call of \texttt{main()} spends most of the time in the
invocation of \texttt{playEM()}. Within this function call, the execution time
is mainly spent within two loops. The first loop (the one from
Listing~\ref{listing:playEM}) simulates the preliminary round. The second loop
simulates the final rounds, from the round of 16 to the final match.

Recall the two main purposes of the trace view, namely identifying hotspots
and estimating load balance. In the case of EMSim, the view indicates the
invocations of \texttt{playGroup()} and \texttt{playFinalRound()} as
significant hotspots. Without considering dependencies, a naive strategy is to
invoke the single calls by individual threads. Besides limited scalability
(there are only 6 calls of \texttt{playGroup()}), this solution results in a
non-optimal load balance since the calls vary in their execution time. However,
the trace view exposes another opportunity for parallelism. Both the group
matches and the final matches lead to distinct calls of
\texttt{playMatchGeneral()} (not shown due to space constraints). Hence, a
second strategy is to use a master-worker pattern for scheduling these calls
with a thread-pool. The next step for the user is to validate the found
parallelization strategies.

The CCT view supports users with the validation of prospective refactorings. 
Figure \ref{fig:cct_view} shows the resulting view for EMSim. Notice that
only relevant call nodes for the investigated parallelization strategies are
shown by spotting them on the trace view. Beginning with the first strategy, we
are interested in possible data dependencies between the invocations of
\texttt{playGroup()} and \texttt{playFinalMatch()}, respectively. The
corresponding analysis (\textit{show deep dependencies}) queries accesses 
to common memory locations among full call paths. It turns out that each
invocation of \texttt{playGroup()} accesses the global variable
\texttt{groupGoals} and each invocation of \texttt{playFinalMatch()} accesses
the stack variables \texttt{goals1} and \texttt{goals2}. Furthermore, the CCT
view enables us to localize these memory accesses in the source code so that
the dependencies can be resolved.

Even though the data dependencies for the first parallelization strategy
can be easily avoided, we focus on the second strategy that puts the calls to
\texttt{playMatchGeneral()} in distinct tasks. Performing the same dependency
analysis as before returns no shared memory accesses between these calls (not
shown due to space constraints). Hence, the user gains the necessary
information to parallelize EMSim by integrating the master-worker pattern. A
student solution achieved a speedup of 6.5 on our 8-core system. Considering
some sequential pre- and post-processing steps, and the control dependencies
between certain matches (e.g., the semi-final has to be played before the
final), the gained speedup confirms Amdahl's law~\cite{amdahl}. These results
conclude that our views are very helpful for parallelization.
