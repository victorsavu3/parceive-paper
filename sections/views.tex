\section{Views}

\subsection{Source View}

The source view is the simplest visualization developed for Parceive. It shows
the highlighted source code in a file that was used to build the instrumented
application.

The usefulness of this view only becomes apparent when it is communicating with
the others presented in this paper. The simplest interaction is focusing and it
allows the source view to pinpoint the definition of functions an loops making
it easy to follow the execution of a program trough the source code.

Hovering is able to provide additional information about entities. For calls it
can indicate where the call originated and for memory references where they
were allocated and referenced.

\subsection{Calling context tree View}

The CCT View is a visualization that allows an user to easily comprehend the
dynamic behavior of an application. It can display and easily navigate Calls,
Loops and References.

The first node present when the view is created is the call to \texttt{main}.
The user can then expand it like any other call by clicking or using the
context menu. When a function is called multiple times the calls are grouped
into a CallGroup to reduce the number of nodes displayed. CallGroups can be
easily decomposed into their Calls by using the context menu.

Loops can be visually identified by icons on nodes. When a loop icon is present
on the right side of a node then that node contains one or more loops. These
loops can be added to the visualization by using the context menu. A icon on
the left side indicates that the call has been made from inside a loop.
Navigating loop executions and loop iterations is similar to calls and allows
the user to see information at any granularity he desires.

CallGroups, Calls, LoopExecutions and LoopIterations are positioned using the
D3 tree layout. Children of a node are always sorted by their start time. This
positioning of nodes creates a clear indicator of the relationship between
them.

References accessed by parts of the application can be displayed by using the
context menu. These nodes are difficult to integrate as part of a graph layout
and are positioned using a force simulation around the rest of the tree.

This view has some additional features that aim to help a developer parallelize
code:

\begin{itemize}
	\item Profiling information is present as the node color
	\item References shared between nodes can be easily identified
	\item It is easy to expand the references accessed during the inclusive
execution of a call or callgroup
	\item A specialized query that only display the shared references between
nodes
\end{itemize}

\section{Performance View}
\label{sec:performance_view}
The performance view provides an interactive visualization of a program's
profiling and event trace data. Its primary purpose is to assist users in
identifying scenarios that could potentially improve performance, by employing
parallel programming as an optimization solution. Tracing and profiling data
provide valuable material for detailed analysis on how a program behaves at
runtime, and presenting a wholistic view of this data in an easy to digest
visualization, can help users quickly comprehend the program, spot potential
performance bottlenecks, and help guide optimization efforts.

Two visualization modes are provided in the performance view: Tracing, and
profiling. The tracing mode presents a hierarchical view of calls made during
the execution of a program. The calls are arranged from left to right
chronologically, and the visibility of the loop information for each call with
a loop execution can be toggled on/off. The trace visualization is useful for
detailed examination of a program, especially where the sequence in which
calls were made is important. The profiling mode presents the user with a
hierarchical view of the functions called during a program's execution. The
length of each function in the view is dictated by the sum of the duration of
calls made to it. The profiling visualization is often sufficient to pinpoint
load imbalance due to problem decomposition and/or identify the origin of
excessive communication time.

As a program grows in size and complexity, the performance data also grows, and
becomes harder to digest all at once. Which is why the performance view
provides call zooming and runtime duration filtering features, to help the user
adjust the view to different data granularity levels. Runtime duration
filtering is a feature that sets the minimum runtime duration required for a
call to be loaded unto the view. The minimum runtime duration value is gotten
by calculating a specified percentage of the runtime duration of the current
top-level call in the view. This feature allows the view render only calls
large enough to be easily visible. With the call zooming feature, a user can
focus the view on a specific call, which then makes it the top-level call,
re-computes a new minimum runtime duration, and loads any child calls of the
focused call that wasn't visible due to the previous runtime duration
constraint.

The view is built and deployed on the visualization framework that is part of
the Parcieve UI. The visualization framework provides a platform for the view
to be delivered to the web, and it also provides an API through which
performance data for a program can be accessed. The performance view is able to
interact with other views on Parceive, by leveraging the event API provided by
the visualization framework. Through the event API, the performance view is
able to broadcast relevant data to other views when calls and loops are hovered
over with the mouse pointer, selected, or zoomed into.
